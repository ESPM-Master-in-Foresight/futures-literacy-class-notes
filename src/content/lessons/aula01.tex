% ========================================
% AULA 1
% ========================================
\pagebreak
\section{Introdução ao Estudo de Futuros}

\begin{slidecontent}
\begin{itemize}[leftmargin=*]
    \item Entender como que o futuro foi tratado na história e como passou a ser tratado no século XX
    \item Entender como chegamos no momento atual em que temos uma área profissional consolidada
\end{itemize}
\end{slidecontent}

\subsection*{Evolução da preocupação com o futuro}


\paragraph{Tempos bíblicos}
A história do Antigo Testamento mostra o quão antiga é a estreita relação entre a posse do poder e o desejo de conhecer o futuro.
O profeta enquanto porta-voz, a figura mais respeita.

\paragraph{Louis-Sébastien Mercier}

Louis-Sébastien Mercier foi um escritor e filósofo francês do século XVIII, conhecido por suas obras visionárias que abordavam o futuro da sociedade. Em seu livro \textit{L'An 2440}, ele descreveu uma utopia futurista, onde a humanidade havia alcançado um estado de paz e prosperidade. Mercier acreditava que a imaginação e a reflexão sobre o futuro eram essenciais para o progresso humano.

Ele via o futuro como um campo aberto para a inovação e a transformação social, incentivando as pessoas a pensar além das limitações do presente. Sua obra influenciou o desenvolvimento do pensamento futurista e destacou a importância de considerar as possibilidades futuras na tomada de decisões no presente.

Mercier foi o primeiro a situar o seu romance utópico num futuro tão longínquo, o ano de 2440 -- distante de nós ainda hoje --, o que demonstra a sua visão avançada sobre o tempo e a evolução da sociedade.

Essa visão otimista do futuro estava na moda no século XVIII, refletindo o espírito iluminista da época, que valorizava a razão, o progresso e a capacidade humana de moldar o futuro.


\paragraph{Como é a descrição do mundo de 2440 feita por Mercier?}
\begin{itemize}
  \item Uma sociedade pacífica e próspera, onde a guerra e a violência foram erradicadas.
  \item Não há escravidão, mas ainda há mornarquia e a pena de morte (embora raramente aplicada).
  \item O trabalho é realizado por apenas algumas horas por dia e o mercantilismo praticamente desapareceu pois não há mais gosto pelo luxo e pelos bens voluptuosos.
  \item O para perdeu todo o seu poder terreno.
  \item A Bastilha acabou, assim como a Sorbonne, pois não se perde mais tempo discutindo sofismas inúteis.
  \item A cidade foi redesenhada com base em princípios científicos, iluminada, tornada saudável e próspera.
  \item Há uma grande palácio de inoculação para tratamento de saúde.
  \item Nas escolas se estuda ciência e línguas, mas não as clássicas, e pouco história.
\end{itemize}

% \noindent\rule{\linewidth}{0.1pt}

\begin{thinkerquote}
  "O tempo presente está grávido do futuro."
  \\ \hfill - G. W. Leibniz
\end{thinkerquote}

\paragraph{Qual o problema desta frase?} Estamos assumindo que o futuro é algo fixo e determinado, que já existe e que só precisamos descobri-lo. Mas será que é assim mesmo?

\subsection*{Conceito implícito de futuro}

\begin{keypoint}
\textbf{Previsão} concretização daquilo que já existe no presente e que apenas aguarda o seu tempo para ser capaz de vir para a luz.
\end{keypoint}

Isso implica que o futuro pode seguir uma trajetória única, ditada pelo progresso humano e técnico, podendo ser extrapolado a partir dos dados concretos que temos hoje. \textbf{A visão de futuro como um mero prolongamento do presente}.

\paragraph{Nicolas de Condorcet}
Condorcet foi um filósofo, matemático e político francês do século XVIII, conhecido por suas contribuições ao pensamento iluminista e à teoria da probabilidade. Ele acreditava no progresso contínuo da humanidade através da razão, da ciência e da educação.
\begin{itemize}
  \item Filosofia da época: história-progresso.
  \item Define-se um futuro possível que só a antiguidade das instituições e costumes existentes impedem de realizar
\end{itemize}


\paragraph{Auguste Comte}
Essa convicção 

% \begin{keypoint}
% \textbf{Conceito Central:} Definição ou explicação de um conceito fundamental abordado nesta aula.
% \end{keypoint}

% \subsection*{Principais Aprendizados}
% \begin{enumerate}[leftmargin=*]
%     \item Aprendizado 1
%     \item Aprendizado 2
%     \item Aprendizado 3
% \end{enumerate}

% \subsection*{Questões para Reflexão}
% \begin{itemize}[leftmargin=*]
%     \item Questão 1?
%     \item Questão 2?
%     \item Questão 3?
% \end{itemize}

\sectionbreak