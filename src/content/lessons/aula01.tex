% ========================================
% AULA 1
% ========================================
\pagebreak
\section{Introdução ao Estudo de Futuros}

\begin{slidecontent}
\begin{itemize}[leftmargin=*]
    \item Entender como que o futuro foi tratado na história e como passou a ser tratado no século XX
    \item Entender como chegamos no momento atual em que temos uma área profissional consolidada
\end{itemize}
\end{slidecontent}

\subsection{Evolução da preocupação com o futuro}


\paragraph{Tempos bíblicos}
A história do Antigo Testamento mostra o quão antiga é a estreita relação entre a posse do poder e o desejo de conhecer o futuro.
O profeta enquanto porta-voz, a figura mais respeita.

\paragraph{Louis-Sébastien Mercier}

Louis-Sébastien Mercier foi um escritor e filósofo francês do século XVIII, conhecido por suas obras visionárias que abordavam o futuro da sociedade. Em seu livro \textit{L'An 2440}, ele descreveu uma utopia futurista, onde a humanidade havia alcançado um estado de paz e prosperidade. Mercier acreditava que a imaginação e a reflexão sobre o futuro eram essenciais para o progresso humano.

Ele via o futuro como um campo aberto para a inovação e a transformação social, incentivando as pessoas a pensar além das limitações do presente. Sua obra influenciou o desenvolvimento do pensamento futurista e destacou a importância de considerar as possibilidades futuras na tomada de decisões no presente.

Mercier foi o primeiro a situar o seu romance utópico num futuro tão longínquo, o ano de 2440 -- distante de nós ainda hoje --, o que demonstra a sua visão avançada sobre o tempo e a evolução da sociedade.

Essa visão otimista do futuro estava na moda no século XVIII, refletindo o espírito iluminista da época, que valorizava a razão, o progresso e a capacidade humana de moldar o futuro.


\paragraph{Como é a descrição do mundo de 2440 feita por Mercier?}
\begin{itemize}
  \item Uma sociedade pacífica e próspera, onde a guerra e a violência foram erradicadas.
  \item Não há escravidão, mas ainda há mornarquia e a pena de morte (embora raramente aplicada).
  \item O trabalho é realizado por apenas algumas horas por dia e o mercantilismo praticamente desapareceu pois não há mais gosto pelo luxo e pelos bens voluptuosos.
  \item O para perdeu todo o seu poder terreno.
  \item A Bastilha acabou, assim como a Sorbonne, pois não se perde mais tempo discutindo sofismas inúteis.
  \item A cidade foi redesenhada com base em princípios científicos, iluminada, tornada saudável e próspera.
  \item Há uma grande palácio de inoculação para tratamento de saúde.
  \item Nas escolas se estuda ciência e línguas, mas não as clássicas, e pouco história.
\end{itemize}

% \noindent\rule{\linewidth}{0.1pt}

\begin{thinkerquote}
  \textit{"O tempo presente está grávido do futuro."}
  \begin{flushright}
    -- G. W. Leibniz
  \end{flushright}
\end{thinkerquote}

\paragraph{Qual o problema desta frase?} Estamos assumindo que o futuro é algo fixo e determinado, que já existe e que só precisamos descobri-lo. Mas será que é assim mesmo?

\subsection*{Conceito implícito de futuro}

\begin{keypoint}
\textbf{Previsão} concretização daquilo que já existe no presente e que apenas aguarda o seu tempo para ser capaz de vir para a luz.
\end{keypoint}

Isso implica que o futuro pode seguir uma trajetória única, ditada pelo progresso humano e técnico, podendo ser extrapolado a partir dos dados concretos que temos hoje. \textbf{A visão de futuro como um mero prolongamento do presente}.

\paragraph{Nicolas de Condorcet}
Nicolas de Condorcet foi um filósofo, matemático e político francês do século XVIII, conhecido por suas contribuições ao pensamento iluminista e à teoria da probabilidade. Ele acreditava no progresso contínuo da humanidade por meio da razão, da ciência e da educação.

Entre 1793 e 1794, Condorcet escreveu o \textit{Esquisse d'un tableau historique des progrès de l'esprit humain}, no qual apresentou uma análise histórica do progresso humano, dividindo-o em dez períodos distintos. Nos primeiros nove períodos, ele descreve o desenvolvimento da humanidade até o ponto de sua própria época, e no décimo período, aborda o futuro, o qual ele imaginava como um estágio de progresso ilimitado.

O \textit{Esquisse} é notável por ser escrito em termos futuros, ou seja, Condorcet antecipa o desenvolvimento da humanidade, com foco na continuidade do progresso. Ele narra a trajetória da humanidade desde seu estágio inicial até a Revolução Francesa, abordando os seguintes períodos:

\begin{itemize}
    \item Povoados primitivos, com pastores e agricultores;
    \item Ascensão da alfabetização;
    \item Grécia Antiga;
    \item Império Romano;
    \item Idade Média;
    \item Renascimento;
    \item Iluminismo;
    \item Revolução Francesa;
    \item Futuro (progresso contínuo e ilimitado).
\end{itemize}

A décima época (pós-revolução Francesa) deveria ser objeto da mais cuidadosa análise científica.
Condorcet já não baseava a sua visão do futuro no Apocalipse, mas nas ciências!

\begin{thinkerquote}
  \textit{“Se o homem pode predizer com uma segurança quase integral os fenômenos dos quais conhece as leis; se, mesmo quando estas lhe são desconhecidas, ele pode, a partir da experiência do passado, prever com uma grande probabilidade os acontecimentos do futuro”.}
  \begin{flushright}
    -- Nicolas de Condorcet
  \end{flushright}
\end{thinkerquote}

Filosofia da época: história-progresso
Define-se um futuro possível que só a antiguidade das instituições e costumes existentes impedem de realizar;
Sugerem a necessidade de uma ruptura radical com o presente.

\begin{thinkerquote}
  \textit{“O tal progresso-história não se reduz a afirmações abstratas sobre a perfectibilidade do gênero humano e não se dispersa em vários "projetos" de reforma [mas] constitui o centro de gravidade da "décima época" e de sua história, a aceleração assumida pelo caminho que até então a humanidade havia percorrido muito lentamente nos caminhos do progresso, ou seja, do aumento da felicidade individual e coletiva. A "décima época" não se apresenta nem como um sonho, nem como um desejo, nem como um "projeto" que defende seu próprio realismo [mas] coloca toda a história humana como fiadora de sua "esperança segura", bem como da "constante e necessária leis » que a regulam. A unidade da exposição histórica da qual faz parte, o fato de esta época seguir-se exatamente como a décima das nove anteriores, significa que o passado, os eventos ocorridos”.}
  \begin{flushright}
    -- Nicolas de Condorcet
  \end{flushright}
\end{thinkerquote}

\paragraph{Auguste Comte}
Essa convicção encontrará plena aplicação no século XIX com o positivismo.
É retomada a fé no progresso do Iluminismo e a transforma em ciência exata

\begin{thinkerquote}
  \textit{“O objetivo de toda a ciência é a previsão”.}
  \begin{flushright}
    -- Auguste Comte
  \end{flushright}
\end{thinkerquote}

Se o astrônomo é capaz de prever com absoluta precisão o estado do sistema solar mesmo com séculos de antecedência, o cientista social também será capaz de prever com precisão o comportamento humano através do estudo do passado. 
Determinar o futuro torna-se assim a tarefa da ciência política exatamente como já é para as ciências empíricas.

É possível interpretar cientificamente dados históricos para elaborar previsões sobre o futuro (preso à ideia de previsão)

\paragraph{Lord Kelvin}
\begin{itemize}
  \item Físico e engenheiro irlandês (1824-1907)
  \item Em 1851: Universo estaria destinado à morte “térmica”
  \item Após obter a formulação moderna da segunda lei da termodinâmica (a degradação inexorável da energia ao longo do tempo) chegou a imaginar o futuro extremo do universo como consequência
  \item Fé na ciência: Ciência garantiria  perfeita compreensão do futuro
\end{itemize}

\subsection{Consolidação da Área de Futuros}
Após a Segunda Guerra Mundial, o estudo sistemático do futuro começou a se consolidar como uma disciplina acadêmica e profissional. Vários fatores contribuíram para esse processo, especialmente a crescente obsessão pelo futuro na sociedade americana.

\subsubsection*{RAND Corporation}
A RAND (Research AND Development) foi um think tank da Defesa, fundado em 1948, com o objetivo de reunir as melhores mentes para projetar novas tecnologias futuristas que pudessem ser aplicadas à defesa dos Estados Unidos. A RAND desempenhou um papel crucial na garantia de uma superioridade tecnológica inigualável para os EUA.

Entre suas funções, destacava-se a produção de guias para os funcionários do governo, com cerca de 300 cientistas de diferentes áreas se dedicando à elaboração de um "quadro do futuro". Esse trabalho ajudou a moldar muitas das políticas militares e de defesa dos Estados Unidos.

\paragraph{Herman Kahn}
Herman Kahn foi um dos mais importantes estrategistas e futuristas da RAND. Seu livro \textit{Thinking About the Unthinkable} (1964) discutiu o conceito do “impensável” — o cenário de uma terceira guerra mundial travada com ogivas nucleares. Este cenário se tornou um foco central da RAND. 

A RAND tinha a liberdade de gerar cenários improváveis, mas não impossíveis. Kahn, em particular, não previu apenas uma terceira guerra mundial, mas também as subsequentes, até a oitava. O mundo futuro imaginado por Kahn estava marcado por uma sucessão de conflitos nucleares, com a racionalização e normalização da guerra. 

Em 1960, os EUA possuíam mais de 17.000 ogivas nucleares, contra pouco mais de mil dos soviéticos. Em 1965, esse número saltou para 30.000. As simulações militares e cenários de Kahn, no entanto, contribuíram pouco para frear essa escalada.

\paragraph{Olaf R. Helmer}
A RAND apostou no desenvolvimento de uma ciência do futuro aplicada à arte militar. Olaf Helmer, matemático alemão, tornou-se o "futurista-chefe" da instituição, com o objetivo de desenvolver uma "teoria geral do futuro". Seu trabalho contribuiu para a formalização do estudo do futuro dentro da área militar.

\paragraph{Robert Jungk}
Em 1954, o jornalista austríaco Robert Jungk publicou um extenso relatório sobre os desenvolvimentos tecnológicos nos Estados Unidos. Seu trabalho incluiu o primeiro relato detalhado do Projeto Manhattan, baseado em entrevistas exclusivas com seus protagonistas. Além disso, Jungk manifestou seu interesse pela crescente obsessão dos Estados Unidos por computadores e outras inovações tecnológicas.

\paragraph{Herbert Goldhamer / Thomas Schelling}
A partir de 1961, Herbert Goldhamer e Thomas Schelling começaram a realizar simulações de guerra mais qualitativas com militares e políticos, incorporando aspectos políticos como as decisões de líderes individuais. Essas simulações tornaram-se uma ferramenta importante para o planejamento militar e para a compreensão dos cenários futuros.

\paragraph{Estado da Arte (1950-60s)}
Os futuristas dos anos 1950 e 1960 compartilhavam alguns pressupostos em comum:
\begin{itemize}
    \item \textbf{Entusiasmo pelas taxas de aceleração do desenvolvimento técnico e científico.}
    \item O passado era visto como um recurso menos relevante para a previsão do futuro.
    \item Era considerada plausível a criação de uma teoria geral do futuro, um sonho positivista de que o futuro poderia ser previsível e moldável.
\end{itemize}

Com isso, surgiu a convicção de que o futuro estava aberto a uma pluralidade de possibilidades. Contudo, os futuristas americanos acabaram adotando uma visão unidirecional da história, que se diferenciava da visão marxista-leninista, que projetava o futuro como uma sociedade comunista. 

A RAND recebeu forte apoio do \textit{Congress for Cultural Freedom} (CCF), uma organização financiada pela Fundação Ford e pela CIA, com o objetivo de promover a ideologia liberal e contrastar a propagação do marxismo nos países ocidentais.

\sectionbreak