% ========================================
% AULA 2
% ========================================
\pagebreak
\section{Causal Layered Analysis (CLA)}

\begin{slidecontent}
Diferenciar os problemas globais que os estudos de futuros têm enfrentado ao longo dos anos.
\end{slidecontent}


\subsection{Incerteza Estrutural}
As ferramentas de previsão nos dão uma estrutura, um caminho do caos do momento para tentar iluminar o que pode acontecer e o que pode acontecer.

Como podemos, graficamente, ilustrar o conceito de Foresight?
Vamos pensar como um evento qualquer como um borrão de tinta.
O núcleo duro, é aquilo que você está vendo, sentindo, experimentando.

\subsection{Futuros e a Importância das Narrativas}
As histórias que contamos frequentemente criam os futuros que vivemos: 
Exemplo clássico: neuroses devido ao um pedido de um(a) parceiro(a) de querer um tempo para relaxar. 
Por que vamos à terapia? 
Para processar as histórias da nossa infância | As histórias dos nossos relacionamentos (profissionais, afetivos, familiares etc.) mais próximos. 
Então, se aceitarmos essa ideia de que nós somos animais contadores de histórias...
Isso significa que tendemos a inventar histórias o tempo todo. 
Essa história pode ser ou não ser verdade, mas ela nos ajuda a dar sentido ao mundo. 
Daí compramos o fato de que juntos, criamos histórias. Se todos nós tivermos comprado uma certa versão, nós vamos continuar vivendo naquela história.
Se escolhemos trocar nossa história, temos que viver numa outra realidade.

\begin{keypoint}
Se quisermos construir \textbf{futuros melhores}, precisamos de \textbf{histórias melhores}.
\end{keypoint}

\subsection{O método CLA}
O método CLA de Foresight é um método que permite construirmos uma narrativa enquanto nós tentamos moldar o futuro!
Isso tudo para trazer a gente para a Análise em Camadas causais (CLA – Causal layered Analysis)
Há várias ilustrações: modelo do iceberg, modelo do oceano, modelo do bolo, modelo da árvore

\subsection{Quatro camadas do CLA}
\begin{itemize}
    \item \textbf{Litany (Litânia):} é a camada mais superficial, é o que está acontecendo, o que está na mídia, o que as pessoas estão falando.
    \item \textbf{Systemic Causes (Causas Sistêmicas):} é a camada que tenta entender as causas sistêmicas, as causas estruturais, as causas institucionais, as causas econômicas, políticas, culturais, sociais.
    \item \textbf{Worldview/Discourse (Visão de Mundo/Discurso):} é a camada que tenta entender as visões de mundo, os paradigmas, as ideologias, as crenças, os valores, as culturas.
    \item \textbf{Myth/Metaphor (Mito/Metáfora):} é a camada mais profunda, é a camada que tenta entender os mitos, as metáforas, as imagens coletivas, os arquétipos.
\end{itemize}

\subsubsection*{Objetivos e Significados do CLA}
\begin{thinkerquote}
    “O objetivo final não é definir o futuro, mas "liberar" o futuro: o ponto fundamental é entender como uma determinada questão é construída, identificada e definida, entendendo quais paradigmas têm sido privilegiados na construção de certas tendências”.
    \begin{flushright}
        -- Joice Preira
    \end{flushright}
\end{thinkerquote}

\begin{thinkerquote}
“A análise em camadas causais (CLA) é oferecida como uma nova teoria e método de pesquisa. Como teoria, procura integrar os modos de conhecimento empirista, interpretativo, crítico e de aprendizagem pela ação. Como método, sua utilidade não está em prever o futuro, mas em criar espaços transformadores para a criação de futuros alternativos. Também é provável que seja útil no desenvolvimento de uma política mais eficaz – mais profunda, inclusiva e de longo prazo” 
\begin{flushright}
    -- (SOHAIL INAYATULLAH)
    \end{flushright}
\end{thinkerquote}

\subsection{Sohail Inayatullah e o pós-estruturalismo}
\begin{table}[h!]
\centering
\renewcommand{\arraystretch}{1.5}
\begin{tabularx}{\textwidth}{>{\bfseries}l X X}
\toprule
\textbf{Aspecto} & \textbf{Estruturalismo} & \textbf{Pós-estruturalismo} \\
\midrule
Objeto de estudo & Estruturas profundas, padrões e sistemas que organizam a realidade social e cultural & Discursos, práticas sociais, relações de poder e as construções mutáveis da realidade \\
\midrule
Visão das estruturas & Fixas, universais, subjacentes à realidade & Fluídas, contingentes, construídas socialmente \\
\midrule
Abordagem do futuro & Baseado em padrões e tendências estruturais & Futuros múltiplos, plurais, abertos e contestados \\
\midrule
Papel do sujeito & Menor ênfase, sujeito como parte da estrutura & Centralidade no sujeito e na subjetividade \\
\midrule
Natureza da verdade & Busca por verdade objetiva e universal & Verdade relativa, situada e discursiva \\
\midrule
Aplicação em Futures & Modelos previsíveis, cenários estruturados & Narrativas múltiplas, construção social do futuro \\
\midrule
Abordagem crítica & Menos crítica às estruturas, tende a aceitar as estruturas como dadas e estáveis & Crítica às estruturas fixas, questiona o poder, a hegemonia e as construções sociais, abrindo espaço para alternativas \\
\bottomrule
\end{tabularx}
\caption{Comparação entre Estruturalismo e Pós-estruturalismo}
\end{table}


\subsection{Exemplos de Aplicação do CLA}





\sectionbreak